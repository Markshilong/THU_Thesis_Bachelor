% !TeX root = ../thuthesis-example.tex

\chapter{结论与展望}

\section{结论}

由于社会经济和科技的迅速发展,近年来摄像头在生活中已经越来越普遍。多摄像头行人位置估计是计算机视觉领域一个重要发展方向,今年来在人工智能和深度学习方法迅速发展的背景下取得了长足进步。在当今摄像头无处不在的背景下,多摄像头行人位置估计的应用场景也越来越多,社会对这项技术的需求也越来越大。这些行人位置估计算法的发展和成熟使得我们可以高效准确地在很多公共场所对行人进行跟踪或轨迹的预测,可以促进自动驾驶技术的进一步发展,甚至还可以应用于电影和游戏行业,促进虚拟现实技术的发展。可见,多摄像头行人位置估计算法很有研究的意义和价值。

本文所提出的多摄像头行人位置估计算法可以分为四个部分:行人检测、位置的相机投影与建模、数据关联和融合、特征学习与输出结果。其中行人检测直接使用即插即用的Alphapose中的预训练网络提取人体姿态节点,从而可以结合外观信息迅速得到行人的足点。此后,利用相机投影的原理将行人足点转为在实际地面上的二维高斯概率分布,从而巧妙地结合了不同视角的摄像机中不同行人位置不同的概率分布信息。在数据关联与融合阶段,利用KL散度作为距离,将地面上的行人二维高斯概率分布利用层次聚类的方式进行融合,再利用融合得到的粗略位置估计结果反向调整前面步骤的参数。这样,将不同视角摄像机的位置概率分布信息进行了很好的融合,有效地降低了遮挡带来的估计效果减退。最后,实现的ResUNet利用了U-Net和残差块的优势,在加深网络深度的同时减少了退化的现象。而在ResUNet基础上实现的ResUNet++则加入了挤压和激励单元、空洞空间金字塔池化和注意力机制,提高了网络的表示性能,同时使得网络能够控制视野、在更重要的特征上投入更多的关注度。ResUNet++在得到的二维高斯概率分布信息上训练,学习到了不同视角摄像机中的行人位置概率分布特征,提升了模型预测效果,接近业界先进水平。

\section{展望}

本文算法可以接近业界先进水平,但由于时间有限,未能继续在此算法中继续拓展和完善来推动其达到领先水平。在将来的工作中,我们还可以在以下方面作新的尝试,从而改进算法、提升效果。首先,目前直接使用的Alphapose需要将高清摄像头输入压缩为$256\times 192$尺寸,因此在这个过程中损失了很多原有信息,未来可以先识别出人物得到其检测框后,将每个行人从高清图中截取下来,分别输入到姿态检测的神经网络中,这样便可以保留更多图像信息,提高每个人足点的预测精度。此外,当前算法,针对同一摄像机而言,行人与摄像机的距离远近对其二维高斯概率分布建模几乎没有影响,但众所周知,近处行人的位置概率分布区域显然应该比远处行人的概率分布更小,近处的行人应该有更精确的位置估计。因此,未来可以加入这一影响参数,也可以加入一个权重系数,使得距离摄像头更近的节点有更高的权重,从而提高位置估计精度。此外,本文算法对行人的外表特征并没有显式地利用,对于视频序列中的时序信息也未加以利用,未来可以加入行人重识别算法的部分功能,同时利用多个连续帧中同一行人位置变化连续的特点,进一步提升神经网络的预测效果。最后,还可以在网络的训练过程中利用数据集增强的方法,如裁剪、加入噪声等方法,进一步扩充数据集,以提高网络的适应性,另外,还可以加入权重衰减(Weight Decay)方法,尝试更多的训练策略,以使得网络发挥其最大性能。可见,此课题仍有很大的发展和探索空间。最后,多摄像头行人位置估计还有很多可以探索的方向,例如,目前大多数多摄像头行人位置估计算法只能在特定数据集上训练和应用,但是实际生活中的场景是非常多样的,因此探索一个可拓展的,即在一个场景训练便可以应用于大多数场景的深度学习网络是很有必要的,此外,当前算法性能相对较差,难以达到实时预测的效果,未来可以从提高算法性能的角度出发,探索更迅速的位置估计算法。